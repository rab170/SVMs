\documentclass{article}
\usepackage{mathtools}
\usepackage{fancyhdr}
\usepackage{extramarks}
\usepackage{amsmath}
\usepackage{amsthm}
\usepackage{amsfonts}
\usepackage{tikz}
\usepackage[plain]{algorithm}
\usepackage{algpseudocode}

\usetikzlibrary{automata,positioning}

%
% Basic Document Settings
%

\topmargin=-0.45in
\evensidemargin=0in
\oddsidemargin=0in
\textwidth=6.5in
\textheight=9.0in
\headsep=0.25in

\linespread{1.1}

\pagestyle{fancy}
\lhead{\hmwkAuthorName}
\chead{\hmwkClass\ (\hmwkClassInstructor): \hmwkTitle}
\rhead{\firstxmark}
\lfoot{\lastxmark}
\cfoot{\thepage}

\renewcommand\headrulewidth{0.4pt}
\renewcommand\footrulewidth{0.4pt}

\setlength\parindent{0pt}

%
% Create Problem Sections
%

\newcommand{\enterProblemHeader}[1]{
    \nobreak\extramarks{}{Problem \arabic{#1} continued on next page\ldots}\nobreak{}
    \nobreak\extramarks{Problem \arabic{#1} (continued)}{Problem \arabic{#1} continued on next page\ldots}\nobreak{}
}

\newcommand{\exitProblemHeader}[1]{
    \nobreak\extramarks{Problem \arabic{#1} (continued)}{Problem \arabic{#1} continued on next page\ldots}\nobreak{}
    \stepcounter{#1}
    \nobreak\extramarks{Problem \arabic{#1}}{}\nobreak{}
}

\newcounter{partCounter}

\newcommand{\hmwkTitle}{Homework 02}
\newcommand{\hmwkDueDate}{February 17, 2016}
\newcommand{\hmwkClass}{Support Vector Machines}
\newcommand{\hmwkClassInstructor}{Dr. Lutz Hamel}
\newcommand{\hmwkAuthorName}{Robert Brown}

%
% Title Page
%

\title{
    \vspace{2in}
    \textmd{\textbf{\hmwkClass}}\\
    \textmd{\textbf{\hmwkTitle}}\\
    \normalsize\vspace{0.1in}\small{Due\ \hmwkDueDate}\\
    \vspace{3in}
}

\author{\textbf{\hmwkAuthorName}}
\date{}

\renewcommand{\part}[1]{\textbf{\large Part \Alph{partCounter}}\stepcounter{partCounter}\\}

%
% Various Helper Commands
%

% Useful for algorithms
\newcommand{\alg}[1]{\textsc{\bfseries \footnotesize #1}}

% For derivatives
\newcommand{\deriv}[1]{\frac{\mathrm{d}}{\mathrm{d}x} (#1)}

% For partial derivatives
\newcommand{\pderiv}[2]{\frac{\partial}{\partial #1} (#2)}

% Integral dx
\newcommand{\dx}{\mathrm{d}x}

% Alias for the Solution section header
\newcommand{\solution}{\textbf{\large Solution}}

% Probability commands: Expectation, Variance, Covariance, Bias
\newcommand{\E}{\mathrm{E}}
\newcommand{\Var}{\mathrm{Var}}
\newcommand{\Cov}{\mathrm{Cov}}
\newcommand{\Bias}{\mathrm{Bias}}

\begin{document}

\maketitle

\pagebreak

\iffalse
\begin{homeworkProblem}
    Give an appropriate positive constant \(c\) such that \(f(n) \leq c \cdot
    g(n)\) for all \(n > 1\).

    \begin{enumerate}
        \item \(f(n) = n^2 + n + 1\), \(g(n) = 2n^3\)
        \item \(f(n) = n\sqrt{n} + n^2\), \(g(n) = n^2\)
        \item \(f(n) = n^2 - n + 1\), \(g(n) = n^2 / 2\)
    \end{enumerate}

    \textbf{Solution}

    We solve each solution algebraically to determine a possible constant
    \(c\).
    \\

    \textbf{Part One}

    \[
        \begin{split}
            n^2 + n + 1 &=
            \\
            &\leq n^2 + n^2 + n^2
            \\
            &= 3n^2
            \\
            &\leq c \cdot 2n^3
        \end{split}
    \]

    Thus a valid \(c\) could be when \(c = 2\).
    \\

    \textbf{Part Two}

    \[
        \begin{split}
            n^2 + n\sqrt{n} &=
            \\
            &= n^2 + n^{3/2}
            \\
            &\leq n^2 + n^{4/2}
            \\
            &= n^2 + n^2
            \\
            &= 2n^2
            \\
            &\leq c \cdot n^2
        \end{split}
    \]

    Thus a valid \(c\) is \(c = 2\).
    \\

    \textbf{Part Three}

    \[
        \begin{split}
            n^2 - n + 1 &=
            \\
            &\leq n^2
            \\
            &\leq c \cdot n^2/2
        \end{split}
    \]

    Thus a valid \(c\) is \(c = 2\).

\end{homeworkProblem}

\pagebreak
\fi

\section{Problem 3.2}
   


\begin{center}
\begin{tabular}{ |c|c|c| } 
\hline
number & property & name \\
\hline
1 & $q(\vec{a}+\vec{b}) = q\vec{a} + q\vec{b}$ & distributivity I \\ 
2 & $(p+q)\vec{a} = p\vec{a} + q\vec{a}$ & distributivity II \\ 
3 & $p(q\vec{a}) = (pq)\vec{a}$ & associativity \\ 
4 & $1\vec{a} = \vec{a}$ & identity \\
\hline
\end{tabular}
\end{center}

1. Consider the $i^{th}$ element of $\vec{a} + \vec{b}$: $q(a_i + b_i) = qa_i + qb_i$. Since scalar multiplication distributes the same $\forall a_i \in \vec{a}$, the property $q(\vec{a}+\vec{b}) = q\vec{a} + q\vec{b}$ follows $\square$ \\

2. consider the $i^{th}$ element of $\vec{a}$: $(p+q)a_i = pa_i + qa_i$. Since scalar multiplication distributes the same $\forall a_i \in \vec{a}$, the property $\vec{a}$, $p\vec{a} + q\vec{a}$ follows $\square$ \\

3. consider the $i^{th}$ element of $\vec{a}$: $p(qa_i) = (pq)a_i$. Since scalar multiplication distributes the same $\forall a_i \in \vec{a}$, the property $p(q\vec{a}) = (pq)\vec{a}$ follows $\square$ \\

4. consider the $i^{th}$ element of $\vec{a}$: $1(a_i) = a_i$.  Since scalar multiplication distributes the same $\forall a_i \in \vec{a}$, the property $1\vec{a} = \vec{a}$ follows $\square$ \\

\section{Problem 3.4}
\begin{center}
 $\vec{w} \cdot \vec{x} = b$ \\
 
 $\vec{w} \cdot \vec{x} - b$ = 0\\
 
 $\smashoperator[r]{\sum_{i=1}^{n}} w_ix_i - b = 0$ \\
 
 $\smashoperator[r]{\sum_{i=1}^{n+1}} w_ix_i = 0 $ \\
 
 
\end{center} 

Where $w_{n+1} = -1$ and $x_{n+1} = b$. Representing our sum as a vector, we are left with a new form $\vec{w} \cdot \vec{x} = 0$ 
that is equivalent to $\vec{w} \cdot \vec{x} = b$, with our bias unit $b$  fixed inside of $\vec{x}$. Now $\vec{w}$ is clearly 
orthogonal to $\vec{x}$ by the definition of orthogonality $(\vec{w} \cdot \vec{x} = 0)$. $\square$.


\end{document}

